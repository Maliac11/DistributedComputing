%%
%% Copyright 2007, 2008, 2009 Elsevier Ltd
%%
%% This file is part of the 'Elsarticle Bundle'.
%% ---------------------------------------------
%%
%% It may be distributed under the conditions of the LaTeX Project Public
%% License, either version 1.2 of this license or (at your option) any
%% later version.  The latest version of this license is in
%%    http://www.latex-project.org/lppl.txt
%% and version 1.2 or later is part of all distributions of LaTeX
%% version 1999/12/01 or later.
%%
%% The list of all files belonging to the 'Elsarticle Bundle' is
%% given in the file `manifest.txt'.
%%

%% Template article for Elsevier's document class `elsarticle'
%% with numbered style bibliographic references
%% SP 2008/03/01
%%
%%
%%
%% $Id: elsarticle-template-num.tex 4 2009-10-24 08:22:58Z rishi $
%%
%%
%%\documentclass[preprint,12pt,3p]{elsarticle}

%% Use the option review to obtain double line spacing
\documentclass[preprint,review,12pt]{cs262}

%% Use the options 1p,twocolumn; 3p; 3p,twocolumn; 5p; or 5p,twocolumn
%% for a journal layout:
%% \documentclass[final,1p,times]{elsarticle}
%% \documentclass[final,1p,times,twocolumn]{elsarticle}
%% \documentclass[final,3p,times]{elsarticle}
%% \documentclass[final,3p,times,twocolumn]{elsarticle}
%% \documentclass[final,5p,times]{elsarticle}
%% \documentclass[final,5p,times,twocolumn]{elsarticle}

%% if you use PostScript figures in your article
%% use the graphics package for simple commands
%% \usepackage{graphics}
%% or use the graphicx package for more complicated commands
%% \usepackage{graphicx}
%% or use the epsfig package if you prefer to use the old commands
%% \usepackage{epsfig}

%% The amssymb package provides various useful mathematical symbols
\usepackage{amssymb}
\usepackage{color}

%% The amsthm package provides extended theorem environments
%% \usepackage{amsthm}

%% The lineno packages adds line numbers. Start line numbering with
%% \begin{linenumbers}, end it with \end{linenumbers}. Or switch it on
%% for the whole article with \linenumbers after \end{frontmatter}.
%% \usepackage{lineno}

%% natbib.sty is loaded by default. However, natbib options can be
%% provided with \biboptions{...} command. Following options are
%% valid:

%%   round  -  round parentheses are used (default)
%%   square -  square brackets are used   [option]
%%   curly  -  curly braces are used      {option}
%%   angle  -  angle brackets are used    <option>
%%   semicolon  -  multiple citations separated by semi-colon
%%   colon  - same as semicolon, an earlier confusion
%%   comma  -  separated by comma
%%   numbers-  selects numerical citations
%%   super  -  numerical citations as superscripts
%%   sort   -  sorts multiple citations according to order in ref. list
%%   sort&compress   -  like sort, but also compresses numerical citations
%%   compress - compresses without sorting
%%
%% \biboptions{comma,round}

% \biboptions{}

\newcommand{\note}[3]{{\color{#2}[#1: #3]}}
%\newcommand{\note}[3]{}%remove comments
\newcommand{\SERENA}[1]{\note{SERENA}{red}{#1}}
\newcommand{\MICHELLE}[1]{\note{MICHELLE}{blue}{#1}}


\journal{CS262: Distributed Systems}

\begin{document}

\begin{frontmatter}

\title{\texttt{ConnectedHearts}: \\ A Distributed System for Viewing the Human Heartbeat}
\tnotetext[label0]{We, Serena Booth and Michelle Cone, affirm our awareness of the standards of the Harvard College Honor Code.}


\author[label1, label0]{Serena Booth}
\address[label1]{sbooth@college.harvard.edu}

\author[label2, label0]{Michelle Cone}
\address[label2]{mcone@college.harvard.edu}

\begin{abstract}
We present a physical distributed system for viewing the human heartbeat. We modified a medicine cabinet by embedding 13 light bulbs around the frame, as well as by replacing the cabinet's glass with one-way mirror. When a person stands in front of the mirror, a webcam hidden behind the mirror measures their pulse. The 13 bulbs, each a virtual machine, run leader election via Bully. The leader starts pulsating with the captured pulse, and then instructs neighboring virtual machines to pulsate as well. Finally, we ensure the synchronization of the bulbs through a distributed gossip algorithm.
\end{abstract}

% \begin{keyword}
% %% keywords here, in the form: keyword \sep keyword
% example \sep \LaTeX \sep template
% %% MSC codes here, in the form: \MSC code \sep code
% %% or \MSC[2008] code \sep code (2000 is the default)
% \end{keyword}

\end{frontmatter}

%%
%% Start line numbering here if you want
%%
% \linenumbers


%% main text
\section{Introduction}
\label{sec1}

\SERENA{TO DO}

\subsection{Artistic Inspiration}

\subsection{As a Distributed System}


\section{Physical Creation}


\section{System Design}

\subsection{Architecture}
\subsection{Bully: Leader Election}
\subsection{Inter-bulb Communication}
\subsection{Realtime Synchronization via Gossip}

\subsubsection{Layout}

As the layout of our system is predetermined, we are able to adjust the gossip algorithm for synchronization in order to ascribe neighboring processes with scores of trustworthiness. As the leader bulb initiates the heartbeat pulsation, the leader is the single most trustworthy bulb. This has implications for all other bulbs, however: as bulbs increase in distance from the leader, a known measure based on the system's fixed layout, their trustworthiness decreases. Thus when a bulb is receiving contradictory gossip from its two neighboring bulbs, it prioritizes the message which was sent by the bulb closer to the leader.  

\section{Synchronization Testing \& Analysis}


\section{Conclusion}



%% The Appendices part is started with the command \appendix;
%% appendix sections are then done as normal sections

%% References
%%
%% Following citation commands can be used in the body text:
%% Usage of \cite is as follows:
%%   \cite{key}         ==>>  [#]
%%   \cite[chap. 2]{key} ==>> [#, chap. 2]
%%

%% References with bibTeX database:

\bibliographystyle{elsarticle-num}
% \bibliographystyle{elsarticle-harv}
% \bibliographystyle{elsarticle-num-names}
% \bibliographystyle{model1a-num-names}
% \bibliographystyle{model1b-num-names}
% \bibliographystyle{model1c-num-names}
% \bibliographystyle{model1-num-names}
% \bibliographystyle{model2-names}
% \bibliographystyle{model3a-num-names}
% \bibliographystyle{model3-num-names}
% \bibliographystyle{model4-names}
% \bibliographystyle{model5-names}
% \bibliographystyle{model6-num-names}

\bibliography{sample}


\end{document}

%%
%% End of file `elsarticle-template-num.tex'.
